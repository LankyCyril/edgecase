\begin{samepage} \begin{table}[h!] \small \begin{tabular}{ll}
\hline
\textbf{Comparison}                                                                     & \textbf{Adjusted p-value} \\
\hline
A subject's reads are closer to each other than to other subjects' reads in the trio    & 7.7e-223                  \\
A subject's reads are closer to each other than to subjects' reads in other populations & <1.0e-300                 \\
Reads within a population are closer to each other than to reads in other populations   & 3.1e-120                  \\
Ashkenazim trio:                                                                        & \textbf{}                 \\
\hspace{.5cm} Father's reads are closer to son's reads than to mother's reads           & 9.5e-33                   \\
\hspace{.5cm} Mother's reads are closer to son's reads than to father's reads           & 1.9e-26                   \\
Chinese trio:                                                                           & \textbf{}                 \\
\hspace{.5cm} Father's reads are closer to son's reads than to mother's reads           & \textit{0.2}\mbox{*}      \\
\hspace{.5cm} Mother's reads are closer to son's reads than to father's reads           & \textit{1.0}\mbox{*}      \\
\hline
\end{tabular}
\caption{
    \small Adjusted \textit{p}-values of the Wilcoxon signed-rank tests on relative Levenshtein distances.
    For each read among all \textit{q} arm telomeric reads,
    closest distances to groups of reads described in the \textit{Comparison} column are compared
    (see \hyperref[sec:methods]{Materials and Methods}).
    \mbox{*}Familial inheritance in the Chinese trio was not significant overall,
    but was detected on several chromosomes (see \textbf{Supplemental Table S3} and \hyperref[sec:discussion]{Discussion}).
}
\label{tab:haptests}
\end{table}
\end{samepage}
