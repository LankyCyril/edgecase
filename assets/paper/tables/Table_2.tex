\begin{samepage} \begin{table}[h!] \small \begin{tabular}{llll}
\hline
\textbf{Telomere} & \textbf{Reference contig}       & \multicolumn{2}{l}{\textbf{Cophenetic correlation}} \\
\textbf{}         & \textbf{}                       & \textbf{r} & \textbf{p}  \\
\hline
2p                &  chr2                           &  0.631     &  6.8e-165   \\
3p                &  3ptel\_1-500K\_1\_12\_12       &  0.607     &  1.4e-235   \\
4p                &  4ptel\_1-500K\_1\_12\_12       &  0.490     &  <1.0e-300  \\
5p                &  chr5                           &  0.760     &  2.4e-194   \\
9p                &  chr9                           &  0.734     &  7.3e-119   \\
12p               &  chr12                          &  0.783     &  2.5e-214   \\
17p               &  17ptel\_1\_500K\_1\_12\_12     &  0.937     &  <1.0e-300  \\
7q                &  chr7                           &  0.838     &  <1.0e-300  \\
8q                &  chr8                           &  0.928     &  <1.0e-300  \\
11q               &  chr11                          &  0.630     &  <1.0e-300  \\
12q               &  chr12                          &  0.881     &  <1.0e-300  \\
14q               &  14qtel\_1-500K\_1\_12\_12\_rc  &  0.842     &  <1.0e-300  \\
15q               &  chr15                          &  0.915     &  <1.0e-300  \\
18q               &  18qtel\_1-500K\_1\_12\_12\_rc  &  0.682     &  <1.0e-300  \\
\hline
\end{tabular}
\caption{
    \small Measures of cophenetic correlation (Pearson's \textit{r} and adjusted \textit{p}-value)
    between the hierarchical clustering and the pairwise distance matrix on each chromosomal arm.
}
\label{tab:cophenetic}
\end{table}
\end{samepage}
